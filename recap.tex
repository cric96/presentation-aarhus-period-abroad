%%%%%%%%%%%%%%%%%%%%%%%%%%%%%%%%%%%%%%%%%%%%%%%%%%%%%%%%%%%%%%%%%%%%%%%%%%%%%%%%
% AMS Beamer series / Bologna FC / Template
% Andrea Omicini
% Alma Mater Studiorum - Università di Bologna
% mailto:andrea.omicini@unibo.it
%%%%%%%%%%%%%%%%%%%%%%%%%%%%%%%%%%%%%%%%%%%%%%%%%%%%%%%%%%%%%%%%%%%%%%%%%%%%%%%%
%\documentclass[handout]{beamer}\mode<handout>{\usetheme{default}}
%
\documentclass[presentation, 9pt]{beamer}\mode<presentation>{\usetheme{AMSBolognaFC}}
%\documentclass[handout]{beamer}\mode<handout>{\usetheme{AMSBolognaFC}}
%%%%%%%%%%%%%%%%%%%%%%%%%%%%%%%%%%%%%%%%%%%%%%%%%%%%%%%%%%%%%%%%%%%%%%%%%%%%%%%%
\usepackage[T1]{fontenc}
\usepackage{wasysym}
\usepackage{amsmath,blkarray}
\usepackage[minted,most]{tcolorbox}
\usepackage{centernot}
\usepackage{fontawesome}
\usepackage{fancyvrb}
\usepackage{minted}
\setminted[scala]{baselinestretch=1,obeytabs=true, tabsize=2}
\usepackage[ddmmyyyy]{datetime}
\definecolor{darkcyan}{rgb}{0.0, 0.55, 0.55}
\renewcommand{\dateseparator}{}
%\renewcommand{\thefootnote}{\fnsymbol{footnote}}
\newcommand{\version}{1}
\usepackage[
	backend=biber,
	citestyle=authoryear-icomp,
	maxcitenames=1,
	bibstyle=numeric]{biblatex}

	\makeatletter

\addbibresource{biblio.bib}
%%%%%%%%%%%%%%%%%%%%%%%%%%%%%%%%%%%%%%%%%%%%%%%%%%%%%%%%%%%%%%%%%%%%%%%%%%%%%%%%
\title[Period Abroad Recap]
{Period Abroad Recap}
%
\subtitle[Graph Neural Networks and Spatio-Temporal tracking]
{Graph Neural Networks and Spatio-Temporal tracking}
%
\author[\sspeaker{Aguzzi}]
{\speaker{Gianluca Aguzzi} \href{mailto:gianluca.aguzzi@unibo.it}{gianluca.aguzzi@unibo.it}}
%
\institute[DISI, Univ.\ Bologna]
{Dipartimento di Informatica -- Scienza e Ingegneria (DISI)\\
\textsc{Alma Mater Studiorum} -- Universit{\`a} di Bologna \\[0.5cm]
}
%
\renewcommand{\dateseparator}{/}
\date[\today]{\today}
%
\AtBeginSection[]
{
  \begin{frame}
  \frametitle{Contents}
  \tableofcontents[currentsubsection, 
	sectionstyle=show/shaded, 
	subsectionstyle=show/shaded]
  \end{frame}
}
\AtBeginSubsection[]
{
  \begin{frame}
  \frametitle{Contents}
  \tableofcontents[currentsubsection, 
	sectionstyle=show/shaded, 
	subsectionstyle=show/shaded]
  \end{frame}
}
%%%%%%%%%%%%%%%%%%%%%%%%%%%%%%%%%%%%%%%%%%%%%%%%%%%%%%%%%%%%%%%%%%%%%%%%%%%%%%%%
\begin{document}
%%%%%%%%%%%%%%%%%%%%%%%%%%%%%%%%%%%%%%%%%%%%%%%%%%%%%%%%%%%%%%%%%%%%%%%%%%%%%%%%

%/////////
\frame{\titlepage}
%/////////
%===============================================================================
\section{Introduction}
\begin{frame}{Activities -- One slide}
\begin{itemize}
	\item \bold{Main idea}: spatio-temporal tracking of a natural phenomenon
	\begin{itemize}
		\item Generation of synthetic data (moving boids)
		\item Studying of spatio-temporal neural network models (forecast)
		\begin{itemize}
			\item Graph Neural Networks
			\item Recurrent Neural Networks
			\item \emph{Recurrent Graph Neural Networks}
		\end{itemize}
		\item Training GNNs through PyTorch and data generated from Alchemist
		\item Integrate Neural networks with Aggregate Computing
		\item Use forecast to compute the best path for nodes
	\end{itemize}
	\item \bold{Clustering}: continue the idea of swarm-based clustering -- avoid the leader election phase
	\begin{itemize}
		\item Inside the cluster, compute the leader (bounded leader election)
	\end{itemize}
\end{itemize}
\begin{alertblock}{Repositories}
	\begin{itemize}
		\item Data generation: \url{https://github.com/cric96/scala-boids}
		\item Alchemist Simulation: \url{https://github.com/cric96/simulation-2022-aarhus-gnn-scafi}
		\item Paper draft: \url{https://github.com/cric96/paper-2022-gnn-ac}
	\end{itemize}	
\end{alertblock}
\end{frame}
%===============================================================================

%===============================================================================
\section{Graph Neural Networks}
\begin{frame}{Artificial Neural Networks -- Recap}
\begin{alertblock}{Definition}
	Collection of units (also called \emph{artificial neuron}) that simulates the behaviour of biological neurons
	connected by weighted edges (also called \emph{synapses}).
\end{alertblock}
\begin{itemize}
	\item Typical organization: \emph{input layer}, \emph{hidden layers}, \emph{output layer}
	\item Each layer is composed of a set of neurons
	\item The output of each layer is computed as: $y = f(\sum_{i=1}^n w_i x_i)$ where $f$ is the activation function
	\item How the neurons are connected is called \emph{topology}
	\item The are several topology types:
	\begin{itemize}
		\item Fully connected (MLP): standard neural network, used for classification and regression problem 
		\item Convolutional: used for image processing
		\item Recurrent: used for time series prediction
		\item \bold{Graph Neural Networks}: used for graph processing
	\end{itemize}
\end{itemize}
\end{frame}
%===============================================================================
\begin{frame}{Graph Neural Networks}
\begin{alertblock}{Definition}
	A graph neural network (GNN)~\scite{scarselli2008graph} is a neural network that operates on \textit{graphs}
\end{alertblock}
\begin{itemize}
	\item \bold{Tasks}
	\begin{itemize}
		\item Node classification: given a graph $G=(V,E)$ and a set of labels $X$, predict the label of each node $v \in V$
		\item Link regression: given a graph $G=(V,E)$ and a set of labels $X$, predict the label of each edge $(u,v) \in E$
		\item Graph classification/regression: given a graph $G=(V,E)$ and a set of labels $X$, predict the label of the graph
	\end{itemize}
	\item \bold{Applications}
	\begin{itemize}
		\item \emph{Representation learning}
		\item \emph{Traffic forecast}~\scite{jiang2022graph}
		\item Physics simulations~\scite{jiang2022graph}
		\item Recommendation systems~\scite{wu2020graph}
		\item Chemistry (Molecular fingerprint, chemical reaction)~\scite{zhou2020graph}
		\item Stock market~\scite{matsunaga2019exploring}
		\item Program verification~\scite{NEURIPS2019_49265d24}
		\item ...
	\end{itemize}
	\item Amazing gentle introduction: 
	\begin{itemize}
		\item \url{https://distill.pub/2021/gnn-intro/}
		\item \url{https://tkipf.github.io/graph-convolutional-networks/}
	\end{itemize}
\end{itemize}
\end{frame}

\begin{frame}{GNN structure \& dynamics}
	\begin{itemize}
		\item \bold{Idea}: iteratively update the node representations 
		by combining the representations of their neighbours and their own representations
		\item \bold{K}: layer of GNNs
		\item \bold{$H^i$}: node representations at layer i ($H^0 = X$)
		\item each layer:
		\begin{itemize}
			\item aggregate the information from the neighbours of each node (\textbf{AGGREGATE} phase) $\sim$ \bold{\lstinline|nbr|}
			\item updates the node representations by combining the aggregated information from neighbours with the current node representations (\textbf{COMBINE}) $\sim$ \bold{\lstinline|foldhood|}
		\end{itemize}
		\item Mathematically:
			$$
			\begin{array}{l}
			a_v^k=\text{AGGREGATE}^k\left\{H_u^{k-1}: u \in N(v)\right\} \\
			H_v^k=\mathbf{COMBINE}^k\left\{H_v^{k-1}, a_v^k\right\},
			\end{array}
			$$
		\item $N(v)$: neighborhood of node $v$
		\item \textbf{AGGREGATE} could be: \textbf{sum}, \textbf{mean}, \textbf{max}, \textbf{min}, \textbf{attention}, etc.
		\item \textbf{COMBINE} could be: \textbf{concatenation}, \textbf{addition}, \textbf{neural network}, etc.
		\item Several architecture are proposed, changing the \textbf{AGGREGATE} and \textbf{COMBINE} functions:
		\begin{itemize}
			\item Graph Convolutional Networks (GCN)~\scite{kipf2016semi}
			\item Graph SAGE~\scite{hamilton2017inductive}
		\end{itemize}
	\end{itemize}
\end{frame}

\begin{frame}[fragile]{GNN structure \& dynamics}
	\centering
	\includegraphics[width=\textwidth]{img/example-gnn.png}

\end{frame}
\begin{frame}{GNN: Aggregate Computing perspective}
	\begin{itemize}
			\item GNN updates the node representations in a \textbf{parallely} {\textcolor{darkcyan}\faThumbsUp} but globally synchronized {\textcolor{bolognafcred}\faThumbsDown}
			\begin{itemize}
				\item Despite that, GNN could be used for distributed computing {\textcolor{darkcyan}\faThumbsUp}
			\end{itemize} 
			\item Stacking several layers of GNNs lead to several hops of information propagation
			\begin{itemize}
				\item Therefore, we could use at most one layer \faArrowRight \, no deep structure for graph representation
			\end{itemize}
			\item If we consider node-wise computation, GNNs processing does not depend on neighbourhood size {\textcolor{darkcyan}\faThumbsUp}
			\item \emph{Learning process}: Centralised Training and Decentralised Execution (CTDE) approach:
			\begin{itemize}
				\item Gather several `snapshots' of the aggregate systems as a graphs
				\begin{itemize}
					\item Features of the nodes: the state of the nodes (sensors, etc.)
					\item Weight of the edges: distances, metrics, neighbourhood data, etc.
				\end{itemize}
				\item Train a GNN on the gathered graphs
				\item `Deploy' the GNN on the aggregate systems
				\begin{itemize}
					\item Each node executes the GNN locally sharing the same parameters (homogenous behaviour)
				\end{itemize}
			\end{itemize}
			\item Potential limitations:
			\begin{itemize}
				\item Synchronized update \faArrowRight \, each node should wait for the update of its neighbours
				\item Dynamic graph: GNN learn structural relation, no temporal relation (e.g., neighbourhood changes, etc.)
			\end{itemize}
		\end{itemize}
\end{frame}
\begin{frame}[fragile]{Towards Timeseries forecast using GNN}
\begin{itemize}
	\item GNN are typically used for static graphs
	\begin{itemize}
		\item connection doesn't change over time
		\item single snapshot of the graph is enough to predict the label
	\end{itemize}
	\item Recent works try to use GNN for time-series prediction \faArrowRight \, spatiotemporal neural networks
	\begin{itemize}
		\item Distributed controller for robots ~\scite{tolstaya2020learning}
		\item Trafic forecast~\scite{zhao2019t}
		\item Weather forecast~\scite{keisler2022forecasting}
	\end{itemize}
	\item Simple Idea: stack GNN structure with a recurrent neural network (RNN) -- or network with memory capabilities
	\item Conceptually, the RNN could be applied in each node performing a local update on its own state
	\begin{itemize}
		\item The state depends on the neighbourhood since it is the output of the GNN
	\end{itemize}	
	\centering
\end{itemize}
\includegraphics[width=\textwidth]{img/idea.pdf}

\end{frame}
%===============================================================================
\section{Aggregate Computing + Neural Networks}
%===============================================================================
\begin{frame}[fragile, allowframebreaks]{Aggregate Computing \faPlus \, Neural Networks: Integration perspective}
\begin{itemize} 
	\item Neural networks can be seen as simple functions
	\begin{itemize}
		\item Some of them should be carefully considered (how to deal with time series? how to deal with graphs?)
	\end{itemize}
	\item In the following, we consider a general schema in how to apply NNs in AC
	\item I don't cover the learning process -- I consider that it is already done using an external process
	\begin{itemize}
		\item \mintinline{scala}|sense("feature")| get the feature of interest
		\item \mintinline{scala}|nbrData| get the associated with the weighted neighborhood
		\item \mintinline{scala}|NN| is the function associated with the neural network
	\end{itemize}
\end{itemize}
\begin{alertblock}{MLP}
No need to handle time or spatial aspects
\begin{minted}{scala}
def main() = NN(sense("feature"))
\end{minted}
\end{alertblock}
\begin{alertblock}{RNN}
Similar to MLP, but you need to use \mintinline{scala}|rep| to encode the temporal aspect
\begin{minted}{scala}
def main() = 
	rep(default){
		memory => NN(sense("feature"), memory)
	}
\end{minted}
\end{alertblock}
\begin{alertblock}{GNN}
	In this case you need to use \mintinline{scala}|nbr| in order to build the local graph-view
\begin{minted}{scala}
def main() = 
	val neighborhoodFeature = includingSelf(nbr(sense("feature")))
	val neighborhoodWeight = excludingSelf(nbrData)
	NN(neighborhoodFeature, neighborhoodWeight)
\end{minted}
NB! The order of the feature vector and the weight should be preserved
\end{alertblock}
\begin{alertblock}{GNN \faPlus \, RNN}
	You need to use \mintinline{scala}|rep| and \mintinline{scala}|nbr| in order to build the local graph-view with memory
\begin{minted}{scala}
def main() =
	rep(default){
		memory => 
			val neighborhoodFeature = includingSelf(nbr(sense("feature")))
			val neighborhoodWeight = excludingSelf(nbrData)
			val nbrMemory = includingSelf(nbr(memory))
			NN(neighborhoodFeature, neighborhoodWeight, nbrMemory)
	}
\end{minted}
NB! Some NNs use as input the neighbourhood memory information
\end{alertblock}
\end{frame}

%===============================================================================
\section{Use case: Spatio-temporal forecast + tracking}
%===============================================================================
\begin{frame}{Use case: spatio temporal forecast + tracking}
\begin{alertblock}{General Idea}
	A Cyber Physical Swarms (CPSWs) are used to track a moving phenomenon. The nodes should:
	\begin{itemize}
		\item Cover the area (coverage) \faArrowRight \, AC
		\item Follow the phenomenon (tracking) \faArrowRight \, AC
		\item Predict the future position of the phenomenon (forecast) in order to improve the tracking \faArrowRight \, GNN + RNN
	\end{itemize}
\end{alertblock}
\begin{alertblock}{Scenario: wildfire tracking}
	A swarm of drone are deployed in a forest to monitor wildfires. 
	Each drone is equipped with a camera and a sensor to measure the temperature. 
	The goal is to track the fire (i.e., moving and following it) and predict its evolution.
\end{alertblock}
\end{frame}
\begin{frame}{Training Process}
\begin{itemize}
	\item We follow a CTDE approach
	\item The data are generated using Alchemist simulations\footnote{\url{https://github.com/cric96/simulation-2022-aarhus-gnn-scafi}}
	\begin{itemize}
		\item It creates a density map from a set of boids that moves in a 2D space using flocking models
		\item A grid of nodes senses the perceived density 
		\item A \emph{snapshot} is captured each \emph{dt} time units \faArrowRight \, it creates a graph in which the labels are related to the density, and the weights are related to the distance (metric)
		\item  Therefore, for each simulation we have a sequence of graphs
	\end{itemize}
	\item The training process is done using a \emph{supervised learning} approach (using PyTorch)
	\begin{itemize}
		\item Given a certain snapshot, the ground truth consists of the next $K$ snapshots
		\item Loss function is the MSE between the predicted and the ground truth
	\item We test several models, focusing on the spatial and temporal aspect
	\begin{itemize}
		\item Does the neighbourhood size affect the performance of the prediction? 
		\item Is the temporal axis enough to get a good prediction?
		\item Is the spatial axis enough to get a good prediction?
	\end{itemize}
	\end{itemize}
	\item \emph{Current result}
	\begin{itemize}
		\item Using only spatial information led to bad predictions
		\item Using only temporal information led to good predictions
		\item The best solution is the one in which we use both spatio and temporal data (with only one hop)
	\end{itemize}

\end{itemize}
\end{frame}
\begin{frame}{Training Process: Class diagram}
\centering
\includegraphics[height=0.9\textheight]{img/structure-training.png}
\end{frame}

\begin{frame}[fragile]{Tracking}
\begin{itemize}
	\item The nodes follow the phenomenon using a window of perception
	\begin{itemize}
		\item A possible AC approach: past perceptions stored in a memory (e.g., a queue):
		\begin{minted}{scala}
		rep(Queue.empty[Double]) { memory => memory :+ sense("perception") }
		\end{minted}
		\item Proposed approach: using forecast produced from \mintinline{scala}{DensityForecast}
	\end{itemize} 
	\item The nodes compute the partial derivative of the density with respect to their local neighbourhood
	\begin{itemize}
		\item Similar to the idea of dense motion field/optical flow~\scite{verri1989motion} in computer vision
		\item It produces a field of delta movements (i.e., a potential field)
	\end{itemize}
	\item The nodes should also maximize the coverage
	\begin{itemize}
		\item they have a force $F$ that is proportional to the neighbourhood \faArrowRight \, if there are too many nodes, they came back to their original position 
	\end{itemize}
\end{itemize}
\centering
{
\includegraphics[width=0.3\textwidth]{img/step-1.png}
\includegraphics[width=0.3\textwidth]{img/step-2.png}
\includegraphics[width=0.3\textwidth]{img/step-3.png}
}
\end{frame}
\begin{frame}{Evaluation}
	\begin{itemize}
		\item \bold{Goal}: verifying the improvement in using the forecast w.r.t. the historical information
		\item \bold{Metrics to consider}:
		\begin{itemize}
			\item \emph{Coverage}: the percentage of nodes that are in the area of interest
			\item \emph{Tracking}: the percentage of nodes that are in the area of interest and are following the phenomenon
		\end{itemize}
		\item we expect that, using forecast, the \emph{Coverage} and the \emph{Tracking} metric should be maximised.
		\item \emph{How to compute the metrics?}
		\begin{itemize}
			\item \emph{Coverage}: each node has a radius of perception, the coverage is the area of interest divided by the intersection of the area covered by the nodes
			\item \emph{Tracking}: compute the density map from the nodes, and compare it with the ground truth
		\end{itemize}
		\item Current status: we are still working on the evaluation process
		\begin{itemize}
			\item The neural networks produced by Python are already compatible with ScaFi (thanks to ScalaPy)
		\end{itemize}
	\end{itemize}
\end{frame}
\begin{frame}{Spatio Temporal Tracking - Activities recap}
\begin{itemize}
	\item[{\textcolor{darkcyan}\faThumbsUp}] Generate synthetic data
	\item[{\textcolor{darkcyan}\faThumbsUp}] Create common interface for several neural network models
	\item[{\textcolor{darkcyan}\faThumbsUp}] Setup training process
	\item[{\textcolor{darkcyan}\faThumbsUp}] Integrating neural network in ScaFi
	\item[{\textcolor{darkcyan}\faThumbsUp}] Setup the simulation 
	\item[\faQuestion] Evaluating the training process with the different models
	\item[\faThumbsDown] Computing the metric in both scenarios (forecast vs historical)
	\item[\faThumbsDown] Comparing the result
	\item[\faQuestion] Set up the simulation using real data (e.g., the one from Toronto?)
\end{itemize}
\end{frame}
%===============================================================================
\section{Clustering}
\begin{frame}{Clustering: draft}
\begin{itemize}
	\item In CPSWs, the nodes are often organized in clusters
	\begin{itemize}
		\item Task-division: each cluster is responsible for a different task (e.g., aggregate processes or predefined cluster as in Buzz)
		\item Partitioning the space: each cluster is responsible for a different area (block S)
		\item Phenomenon tracking: clusters follow a certain environment phenomenon	(swarm-based clustering~\scite{})
	\end{itemize}
	\item In order to form areas, we typically leverage a leader election process:
	\begin{itemize}
		\item In swarm clustering: the centre (leader) depends on a local minimum
		\item In S: the leader is elected using the grain and a symmetry-breaking process
	\end{itemize}
	\item Another approach: try to form a cluster without a centre
	\begin{itemize}
		\item The leader process could be performed inside the cluster
	\end{itemize}
	\item preliminary idea: using a majority voting role \faPlus \, random colouring process
	\begin{itemize}
		\item The colour (i.e., the cluster ID) depends on the majority of the neighbour's colour. If there is no majority, the colour is randomly chosen
		\item Benefit: the cluster doesn't depend on a node but is continuously updated but the elements of the cluster
	\end{itemize}
	\centering
	\includegraphics[width=0.6\textwidth]{img/clustering.png}
\end{itemize}
\end{frame}
%===============================================================================

%===============================================================================
\section*{}
%===============================================================================

%/////////
\frame{\titlepage}
%/////////

%===============================================================================
\section*{\refname}
%===============================================================================

%%%%
\setbeamertemplate{page number in head/foot}{}
%/////////
\begin{frame}[c,noframenumbering, allowframebreaks]{\refname}
%\begin{frame}[t,allowframebreaks,noframenumbering]{\refname}
	\tiny
	\printbibliography
\end{frame}
%/////////

%%%%%%%%%%%%%%%%%%%%%%%%%%%%%%%%%%%%%%%%%%%%%%%%%%%%%%%%%%%%%%%%%%%%%%%%%%%%%%%%
\end{document}
%%%%%%%%%%%%%%%%%%%%%%%%%%%%%%%%%%%%%%%%%%%%%%%%%%%%%%%%%%%%%%%%%%%%%%%%%%%%%%%%
